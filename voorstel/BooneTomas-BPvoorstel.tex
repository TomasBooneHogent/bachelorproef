%==============================================================================
% Sjabloon onderzoeksvoorstel bachproef
%==============================================================================
% Gebaseerd op document class `hogent-article'
% zie <https://github.com/HoGentTIN/latex-hogent-article>

% Voor een voorstel in het Engels: voeg de documentclass-optie [english] toe.
% Let op: kan enkel na toestemming van de bachelorproefcoördinator!
\documentclass{hogent-article}

% Invoegen bibliografiebestand
\addbibresource{voorstel.bib}

% Informatie over de opleiding, het vak en soort opdracht
\studyprogramme{Professionele bachelor toegepaste informatica}
\course{Bachelorproef}
\assignmenttype{Onderzoeksvoorstel}
% Voor een voorstel in het Engels, haal de volgende 3 regels uit commentaar
% \studyprogramme{Bachelor of applied information technology}
% \course{Bachelor thesis}
% \assignmenttype{Research proposal}

\academicyear{2024-2025} % TODO: pas het academiejaar aan

% TODO: Werktitel
\title{AI video classificatie van bendevorming}

% TODO: Studentnaam en emailadres invullen
\author{Tomas Boone}
\email{tomas.boone@student.hogent.be}

% TODO: Medestudent
% Gaat het om een bachelorproef in samenwerking met een student in een andere
% opleiding? Geef dan de naam en emailadres hier
% \author{Yasmine Alaoui (naam opleiding)}
% \email{yasmine.alaoui@student.hogent.be}

% TODO: Geef de co-promotor op
\supervisor[Co-promotor]{S. Beekman (Synalco, \href{mailto:sigrid.beekman@synalco.be}{sigrid.beekman@synalco.be})}

% Binnen welke specialisatierichting uit 3TI situeert dit onderzoek zich?
% Kies uit deze lijst:
%
% - Mobile \& Enterprise development
% - AI \& Data Engineering
% - Functional \& Business Analysis
% - System \& Network Administrator
% - Mainframe Expert
% - Als het onderzoek niet past binnen een van deze domeinen specifieer je deze
%   zelf
%
\specialisation{AI \& Data Engineering}
\keywords{google cloud platform, video classification, video streaming, rtsp, inference on the cloud, video inference}

\begin{document}

\begin{abstract}
  Video classificatie is een interessant domein van machine vision dat professioneel nog weinig daglicht ziet.
  Veel data van camera's kan niet in real time geanalyseerd worden en wordt meestal pas bekeken na de feiten.
  Het lijkt mij daarom interessant om deze data te nuttigen voor classificatie of herkenning van bepaalde scenes.
  Tijdens een gesprek met politie Gent (een van de grotere camera gebruikers in Belgie) vertelden ze mij in welke use cases ze geinteresseerd
  zouden zijn. Waarvan de belangrijkste bendevorming is.

\end{abstract}

\tableofcontents

% De hoofdtekst van het voorstel zit in een apart bestand, zodat het makkelijk
% kan opgenomen worden in de bijlagen van de bachelorproef zelf.
%---------- Inleiding ---------------------------------------------------------

\section{Introductie}%
\label{sec:introductie}

België telt zo een 50.000 bewakingscamera's volgens de privacy comissie (geen echte bron, ruwe schatting), maar zij beweren dat dit een understatement van 50% is.
Een groot deel daarvan zijn eigendom van de politie. Deze worden voornamelijk gebruikt om na de feiten specifieke timeframes te onderzoeken.
Na een eerste communicatie blijkt er interesse van hun kant om in real time video analyse van deze bronnen te bekijken.
%---------- Stand van zaken ---------------------------------------------------

\section{State-of-the-art}%
\label{sec:state-of-the-art}

Video analyse in AI termen betekent dat een CNN afbeelding per afbeelding bekijkt alsook de semantische verschillen tussen opeenvolgende frames.
Aangezien video voornamelijk in 24 of 25 fps wordt opgenomen betekent dat foto's heel snel behandeld moeten worden. En dit is slechts voor één camera.
Ongelimiteerde rekenkracht en schaalbaarheid zijn hier dus de belangrijkste factoren. Daarom koos ik voor een oplossing in een cloud instantie.

Natuurlijk zijn er GDPR afwegingen en is er (veelal onterecht) algemeen wantrouwen in cloud computing. Maar deze laat ik hier ter zijde.

%---------- Methodologie ------------------------------------------------------
\section{Methodologie}%
\label{sec:methodologie}

Het doel is om een professioneel werkend resultaat met voldoende performantie en schaalbaarheid te bekomen. Aangezien mijn GCP en (professioneel) data engineering kennis beperkt is,
ga ik zoveel mogelijk proberen gebruik maken van bestaande producten. Eenvoud is samen met performantie en schaalbaarheid de belangrijkste NFR.

Na een kort onderzoek leek het mij duidelijk om AutoML modellen te gebruiken van Google Vertex AI. Dit zijn AI modellen volledig gemanaged door Google.
Het kan eventueel interessant zijn om de resultaten te vergelijken met een custom model.

Het streamen van video feeds naar een cloud instantie zal waarschijnlijk de crux van het project vormen. Het is mij nog niet duidelijk welke manieren er bestaan om video met een voldoende kleine vertraging over het internet te streamen.
Compressie en chunking (netwerking) zullen hierbij onder de loep moeten genomen worden.

Het trainen en uiteindelijk functioneren van een CNN hangt in grote mate af van de kwaliteit van de dataset. Hier verwacht ik problemen om voldoene kwaliteitsvolle data te verzamelen en labellen.
Gevoelige data (tijdelijk) opslaan op de Cloud blijft een even gevoelig onderwerp.
%---------- Verwachte resultaten ----------------------------------------------
\section{Verwacht resultaat, conclusie}%
\label{sec:verwachte_resultaten}

Ik kan het project slechts geslaagd noemen als er een accuraatheid van 97% behaald wordt. Ook moet er binnen een redelijk timeframe een predictie geretourneerd worden.
Deze 2 factoren zullen het uiteindelijke succes van het product bepalen.

\printbibliography[heading=bibintoc]

\end{document}