%---------- Inleiding ---------------------------------------------------------

\section{Introductie}%
\label{sec:introductie}

België telt zo een 50.000 bewakingscamera's volgens de privacy comissie (geen echte bron, ruwe schatting), maar zij beweren dat dit een understatement van 50% is.
Een groot deel daarvan zijn eigendom van de politie. Deze worden voornamelijk gebruikt om na de feiten specifieke timeframes te onderzoeken.
Na een eerste communicatie blijkt er interesse van hun kant om in real time video analyse van deze bronnen te bekijken, voornamelijk om bende vorming te detecteren.
%---------- Stand van zaken ---------------------------------------------------

\section{State-of-the-art}%
\label{sec:state-of-the-art}

Video analyse in AI termen betekent dat een CNN afbeelding per afbeelding bekijkt alsook de semantische verschillen tussen opeenvolgende frames.
Aangezien video voornamelijk in 24 of 25 fps wordt opgenomen betekent dat foto's heel snel behandeld moeten worden. En dit is slechts voor één camera.
Ongelimiteerde rekenkracht en schaalbaarheid zijn hier dus de belangrijkste factoren. Daarom koos ik voor een oplossing in een cloud instantie.

Natuurlijk zijn er GDPR afwegingen en is er (veelal onterecht) algemeen wantrouwen in cloud computing. Maar deze laat ik hier ter zijde.

%---------- Methodologie ------------------------------------------------------
\section{Methodologie}%
\label{sec:methodologie}

Het doel is om een professioneel werkend resultaat met voldoende performantie en schaalbaarheid te bekomen. Aangezien mijn GCP en (professioneel) data engineering kennis beperkt is,
ga ik zoveel mogelijk proberen gebruik maken van bestaande producten. Eenvoud is samen met performantie en schaalbaarheid de belangrijkste NFR.

Na een kort onderzoek leek het mij duidelijk om AutoML modellen te gebruiken van Google Vertex AI. Dit zijn AI modellen volledig gemanaged door Google.
Het kan eventueel interessant zijn om de resultaten te vergelijken met een custom model.

Het trainen en uiteindelijk functioneren van een CNN hangt in grote mate af van de kwaliteit van de dataset. Hier verwacht ik problemen om voldoene kwaliteitsvolle data te verzamelen en labellen.
Gevoelige data (tijdelijk) opslaan op de Cloud blijft een even gevoelig onderwerp.
%---------- Verwachte resultaten ----------------------------------------------
\section{Verwacht resultaat, conclusie}%
\label{sec:verwachte_resultaten}

Ik kan het project slechts geslaagd noemen als er een accuraatheid van 97% behaald wordt. Ook moet er binnen een redelijk timeframe een predictie geretourneerd worden.
Deze 2 factoren zullen het uiteindelijke succes van het product bepalen.