%---------- Inleiding ---------------------------------------------------------

% TODO: Is dit voorstel gebaseerd op een paper van Research Methods die je
% vorig jaar hebt ingediend? Heb je daarbij eventueel samengewerkt met een
% andere student?
% Zo ja, haal dan de tekst hieronder uit commentaar en pas aan.

%\paragraph{Opmerking}

% Dit voorstel is gebaseerd op het onderzoeksvoorstel dat werd geschreven in het
% kader van het vak Research Methods dat ik (vorig/dit) academiejaar heb
% uitgewerkt (met medesturent VOORNAAM NAAM als mede-auteur).
% 

\section{Inleiding}%
\label{sec:inleiding}


Bij het toekennen van zakelijke leningen moeten banken en kredietverstrekkers een nauwkeurige inschatting maken van het risico op wanbetaling. Momenteel baseren analisten zich voornamelijk op de meest recente jaarrekening van de specifieke onderneming, aangevuld met externe credit scores. Deze aanpak is echter reactief en isoleert het bedrijf vaak van zijn bredere sectorale context. Bovendien is het manueel vergelijken van een onderneming met al haar sectorgenoten een onbegonnen werk zonder de juiste technologische ondersteuning.

Dit onderzoek kan interessant zijn voor financiële risicoanalisten en beleidsmakers binnen banken die verantwoordelijk zijn voor het bepalen van leningsvoorwaarden (zoals rentevoeten en looptijden).

De huidige methoden voor risicoanalyse maken onvoldoende gebruik van de enorme hoeveelheid beschikbare historische data, waardoor sector-brede trends en vroegtijdige waarschuwingssignalen voor falingen over het hoofd worden gezien.

Dit leidt tot de centrale onderzoeksvraag:
\textit{In welke mate kan een Big Data-architectuur op basis van publieke NBB-data bijdragen tot een nauwkeurigere financiële risico-inschatting en voorspelling van solvabiliteit bij kredietverlening?}

De onderzoeksdoelstelling is het ontwikkelen van een Proof of Concept (PoC) waarbij de volledige dataset van de Nationale Bank van België (alle neerleggingen van de afgelopen drie jaar) wordt ingeladen in een schaalbare database (Google BigQuery). Hiermee wordt dan gëxperimenteerd en getest indien sector-brede benchmarking en voorspellende analyses een meerwaarde kunnen bieden bovenop de klassieke dossierbehandeling.

%---------- Stand van zaken ---------------------------------------------------

\section{Literatuurstudie}%
\label{sec:literatuurstudie}

H
De financiële gezondheid van een onderneming wordt traditioneel geëvalueerd aan de hand van ratio-analyse (zoals liquiditeit, solvabiliteit en rentabiliteit) op basis van de jaarrekening~\autocite{Ooghe2020}. Er bestaan diverse modellen, zoals de Altman Z-score, die op basis van deze ratio's de kans op faillissement trachten te voorspellen.

Echter, met de opkomst van 'FinTech' verschuift de focus van statische analyse naar dynamische Big Data-toepassingen. Vakliteratuur toont aan dat het gebruik van cloud-native datawarehouses, zoals Google BigQuery of Amazon Redshift, het mogelijk maakt om datasets van miljoenen rijen in seconden te verwerken~\autocite{Tigani2014}. Hierdoor wordt het mogelijk om niet enkel naar één bedrijf te kijken, maar de prestaties direct af te zetten tegenover *real-time* berekende sectorgemiddelden.

Daarnaast is er in het domein van Machine Learning veel onderzoek naar het voorspellen van tijdreeksen (Time Series Forecasting). Het toepassen van dergelijke algoritmen op financiële data (jaarrekeningen over meerdere jaren) kan patronen blootleggen die voor het menselijk oog verborgen blijven. Er is echter nog ruimte voor onderzoek naar de specifieke toepassing hiervan op de Belgische context met de specifieke structuur van de NBB-dataset.



%---------- Methodologie ------------------------------------------------------
\section{Methodologie}%
\label{sec:methodologie}


Het onderzoek zal verlopen in vier fasen:

\begin{enumerate}
    \item \textbf{Data Ingestie \& Engineering:} De data van de Nationale Bank van België wordt verkregen via hun Open Data portaal (doorgaans in XML of CSV formaat). Er wordt een ETL-pipeline (Extract, Transform, Load) opgezet om deze data, betreffende alle Belgische ondernemingen over de laatste drie boekjaren, op te schonen en in te laden in Google BigQuery.
    \item \textbf{Beschrijvende Analyse:} Via SQL-queries worden sectorgemiddelden berekend op basis van NACE-codes. Er wordt gekeken naar spreiding en clustering van ondernemingen om 'gezonde' versus 'ongezonde' profielen te definiëren.
    \item \textbf{Voorspellende Analyse:} Er wordt een experiment opgezet met een Machine Learning model (bijvoorbeeld via BigQuery ML of Python/scikit-learn) om op basis van de historie van jaar $X-2$ en $X-1$ de resultaten van jaar $X$ te voorspellen. De accuraatheid van dit model wordt getoetst.
    \item \textbf{Evaluatie:} De resultaten worden gevisualiseerd in een dashboard (bv. Looker Studio) om de bruikbaarheid voor een risicoanalist te evalueren. Kan de tool sneller een correct risicoprofiel schetsen dan een manuele analyse?
\end{enumerate}

De benodigde tools zijn Google Cloud Platform (Storage, BigQuery), Python (voor data manipulatie) en visualisatiesoftware.

Tijdschatting:
\begin{itemize}
    \item Maand 1: Literatuurstudie en data-acquisitie NBB.
    \item Maand 2: Opzet BigQuery en ETL-pipeline.
    \item Maand 3: Data-analyse en ontwikkeling voorspellend model.
    \item Maand 4: Rapportage, visualisatie en conclusie.
\end{itemize}

%---------- Verwachte resultaten ----------------------------------------------
\section{Verwacht resultaat, conclusie}%
\label{sec:verwachte_resultaten}


Het verwachte eindresultaat is een werkende Proof of Concept van een risico-dashboard.

Concreet verwacht ik:
\begin{itemize}
    \item Een gevulde BigQuery database die performant queries kan uitvoeren op de volledige Belgische bedrijfsdataset.
    \item Inzicht in sector-specifieke trends: bijvoorbeeld, "Hoe presteert de horeca in West-Vlaanderen ten opzichte van het nationale gemiddelde?".
    \item Een evaluatie van de haalbaarheid om jaarrekeningen te voorspellen: is de historische data van de NBB voldoende voorspellend voor de toekomst, of zijn er te veel externe factoren?
\end{itemize}

De meerwaarde voor de banksector is evident: door risico's beter in te schatten, kunnen leningen met meer precisie worden toegekend. Dit betekent lagere rentes voor gezonde bedrijven en bescherming voor de bank tegen 'bad debt' bij risicovolle ondernemingen. Indien het voorspellende model accuraat blijkt, kan dit het acceptatieproces voor leningen deels automatiseren.
